\section{并行计算实验}
在本章中,本文将完成并行计算数据实验,对实验结果进行分析,评估并行
算法的性能和效果,为进一步的研究和应用提供参考和指导。
\subsection{实验目的}
本研究的实验旨在基于多进程与多线程相结合的策略,对 Epi-Quant 工具进行并行化优化,以解决其在处理大规模细胞图像数据时存在的效率瓶颈问题。

本实验的核心目标是通过将任务细化为可并发执行的单元,并在不同层级引入多线程(如在特征提取或图像处理中的像素级并行)与多进程(如在多个细胞区域或图像帧之间的独立计算)相结合的调度机制,充分利用多核 CPU 资源,加快处理速度。在此基础上,实验将系统评估优化前后的性能差异,重点考察任务运行时间,验证并行机制对整体流程的加速比表现。
\subsection{实验环境}
表5-1是基于DirectX诊断工具给出的系统参数以及Intel官网给出的规格参数:
\begin{table}[htbp]
    \centering
    \caption{Intel Core i9-13900H 处理器规格}
    \label{tab:i9_specs}
    \begin{tabular}{ll}
        \toprule
        \textbf{参数} & \textbf{规格说明} \\
        \midrule
        核心总数 & 14 \\
        性能核(P-core)数量 & 6 \\
        能效核(E-core)数量 & 8 \\
        线程总数 & 20 \\
        最大睿频频率 & 5.40 GHz \\
        睿频加速 Max 技术 3.0 频率 & 5.40 GHz \\
        性能核最大睿频频率 & 5.40 GHz \\
        能效核最大睿频频率 & 4.10 GHz \\
        缓存 & 24 MB Intel\textsuperscript{\textregistered} Smart Cache \\
        处理器基础功耗 & 45 W \\
        最大睿频功耗 & 115 W \\
        最小保证功率 & 35 W \\
        \bottomrule
    \end{tabular}
\end{table}
\subsection{实验结果}
\subsubsection{细胞特征提取并行化结果}
将并行化的程序在Python平台运行5次,取平均值,记录为下表:
\begin{table}[!htbp]
    \centering
    \caption{并行化细胞特征提取平均耗时统计}
    \begin{tabular*}{350pt}{@{\extracolsep{\fill}}cc}
        \toprule
        步骤               & 平均耗时(秒) \\
        \midrule
        文件排序时间       & 0.0002 \\
        加载时间           & 0.7621 \\
        处理时间           & 9.7935 \\
        UI 更新时间         & 0.1829 \\
        DataFrame 合并时间 & 0.0384 \\
        Excel 保存时间     & 0.7020 \\
        总时间             & 12.0433 \\
        \bottomrule
    \end{tabular*}
\end{table}

已知并行效率的计算公式\cite{ref12}\cite{ref13}为:
\[
\text{加速比} = \frac{\text{串行时间}}{\text{并行时间}} \tag{5-1} 
\]

% 并行效率公式(通用)
\[
\text{并行效率} = \frac{\text{加速比}}{\text{线程/进程数}} \tag{5-2} 
\]

将原代码所记录的时间记录为串行时间,由(5-1)和(5-2)计算得出:
\begin{table}[htbp]
    \centering
    \caption{加速比与并行效率}
    \begin{tabular}{lcc}
    \toprule
    指标 &   结果 \\
    \midrule
    加速比(Speedup)  & 2.179 \\
    并行效率(Efficiency)  & 1.089 \\
    \bottomrule
    \end{tabular}
    \end{table}
\newpage
\subsubsection{细胞分类并行化结果}
将并行化的程序在Python平台运行5次,取平均值,记录为下表:

\begin{table}[!htbp]
    \centering
    \caption{并行化细胞分类平均耗时统计}
    \begin{tabular*}{300pt}{@{\extracolsep{\fill}}cc}
        \toprule
        步骤             & 耗时(秒) \\
        \midrule
        PCA 执行耗时     & 1.6210     \\
        K-means 执行耗时 & 1.6701     \\
        图像处理耗时     & 28.4078    \\
        总执行耗时       & 31.7604    \\
        \bottomrule
    \end{tabular*}
\end{table}


将原代码所记录的时间记录为串行时间,由(5-1)和(5-2)计算得出:
\begin{table}[htbp]
    \centering
    \caption{加速比与并行效率}
    \begin{tabular}{lcc}
    \toprule
    指标  & 结果 \\
    \midrule
    加速比(Speedup) & 1.51 \\
    并行效率(Efficiency)  & 0.755 \\
    \bottomrule
    \end{tabular}
\end{table}
\subsubsection{细胞追踪并行化结果}
将并行化的程序在Python平台运行5次,取平均值,记录为下表:
\begin{table}[htbp]
    \centering
    \caption{并行化细胞追踪平均耗时统计}
    \begin{tabular*}{350pt}{@{\extracolsep{\fill}}lcc}
        \toprule
        步骤 & 时间(秒) & 占比(\%) \\
        \midrule
        追踪所有细胞     & 46.5262 & 60.24 \\
        保存追踪数据     & 0.0769  & 0.10 \\
        生成对比图像     & 30.6142 & 39.64 \\
        总执行耗时       & 77.2173 & 100.00 \\
        \bottomrule
    \end{tabular*}
\end{table}

将原代码所记录的时间记录为串行时间,由(5-1)和(5-2)计算得出:
\begin{table}[htbp]
    \centering
    \caption{加速比与并行效率}
    \begin{tabular}{lcc}
    \toprule
    指标  & 结果 \\
    \midrule
    加速比(Speedup) & 1.401 \\
    并行效率(Efficiency)  & 0.701 \\
    \bottomrule
    \end{tabular}
\end{table}
\newpage
\subsection{结果分析}
将三个部分的并行效果进行汇总,记录为下表:
\begin{table}[htbp]
    \centering
    \caption{细胞图像处理各阶段加速比与并行效率汇总}
    \begin{tabular}{lccc}
    \toprule
    阶段 & 串行时间(秒) & 并行时间(秒) & 加速比 / 并行效率 \\
    \midrule
    特征提取 & 26.2361 & 12.0433 & 2.179 / 1.089 \\
    细胞分类 & 48.0328 & 31.7604 & 1.512 / 0.756 \\
    细胞追踪 & 108.2348 & 77.2173 & 1.401 / 0.701 \\
    \bottomrule
    \end{tabular}
\end{table}
从数据中可以看出:

细胞特征提取阶段的并行优化效果最为显著。串行执行时间为 26.2361 秒,并行优化后降为 12.0433 秒,对应加速比为 2.179,效率为 1.089。这一结果甚至略超线性加速,可能得益于任务划分精细、内存访问局部性良好,以及多线程带来的缓存命中率提升。说明该阶段具备高度可并行性,且资源调度效率较高。

细胞分类阶段的加速效果相对一般,串行时间为 48.0328 秒,并行时间为 31.7604 秒,加速比 1.512,并行效率 0.756。该阶段涉及的 PCA 和 K-means 操作具有一定计算强度,但部分处理过程可能存在串行瓶颈,影响整体加速效果。任务划分和线程负载均衡仍有优化空间。

在细胞追踪阶段,通过对帧间匹配任务进行并行化处理,使总耗时由串行的 108.2348 秒减少至并行的 77.2173 秒,获得了 1.401 的加速比和 0.701 的并行效率。该阶段的性能提升主要来源于对不同帧之间细胞轮廓比对的并行处理,显著压缩了核心计算时间。然而,由于帧与帧之间仍存在一定的数据依赖关系,限制了并行深度,加之其他步骤(如图像生成)未实现并行,整体加速效果受到制约。因此,尽管当前优化取得了一定成效,仍有进一步提升并行效率的空间。

\subsection{实验结论与展望}
本研究采用了并行计算对 Epi-Quant 工具包中的细胞特征提取、细胞分类和细胞追踪等任务进行了优化。实验结果表明,细胞特征提取阶段通过并行化处理显著提升了计算效率,尤其在特征计算步骤中取得了显著加速。这表明特征提取的计算任务具有较强的并行化潜力,能够有效提高整体处理速度。

在细胞分类阶段,尽管并行计算提升了部分计算效率,但由于 PCA 和 K-means 算法的迭代过程较为复杂,导致加速效果不如细胞特征提取阶段明显。这也表明,对于一些依赖性较强的计算任务,虽然并行化能加速部分处理过程,但整体效果仍有改进的空间。

在细胞追踪阶段,通过并行计算优化了帧间匹配的过程。尽管存在较大的数据依赖性,合理的线程分配依然提高了计算速度,并增强了处理大规模图像数据的能力。尽管加速比有限,但并行计算在稳定性和效率提升方面仍然发挥了重要作用。

未来的工作可以考虑进一步优化并行化策略,特别是针对迭代依赖较强的算法,可能需要改进算法或采用更高效的并行方法。随着细胞图像分析规模的扩大,探索更高效的硬件加速方法(如 GPU 或分布式计算)也是提高计算效率的重要方向。此外,结合智能资源调度和任务划分策略,进一步提升系统的整体性能,预期能为大规模细胞图像数据的处理提供更高效的解决方案。






