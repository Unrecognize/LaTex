\section{引言}

\subsection{课题研究背景及意义}

近年来,随着活细胞成像与单细胞测序等技术的迅速发展,海量的细胞图像数据不断涌现,对图像分析工具的性能提出了更高要求。同时涌现出很多基于深度学习算法的细胞图像处理工具\cite{ref1},针对上皮细胞的量化分析,Epi-Quant 凭借结合 Cellpose 的高效分割能力、自主开发的基于主成分分析(PCA)与 K-means 聚类的细胞分类模块,以及基于差异度矩阵与匈牙利算法的追踪方法,成为细胞动力学研究中的重要工具\cite{ref2}。Epi-Quant 能够对形态异质性的上皮细胞群体进行精准分类,并支持多组别的定量比较分析,在细胞发育、再生医学、疾病模型研究中具有广泛应用前景。

随着图像数据规模的不断扩大,Epi-Quant 在处理大规模数据集时,仍存在处理速度有限、计算资源利用率不足等问题。这不仅制约了其在高通量实验数据分析中的应用效率,也影响了对细胞行为细微动态变化的实时捕捉与深入解析。已有研究表明,传统串行处理方式在处理百万级细胞图像数据时,难以满足实时分析与交互式操作的需求。

并行计算技术,尤其是多核 CPU 并行(例如OpenMP, multiprocessing)与 GPU 加速(例如CUDA, TensorRT)技术\cite{ref3}\cite{ref4},已广泛应用于图像处理、深度学习和生物图像分析领域,能够显著提升程序执行速度与计算资源利用率。在细胞图像领域,Cellpose、DeepCell 等工具已经展示了 GPU 加速在大规模图像分割任务中的巨大潜力\cite{ref5}。

将并行技术系统性引入 Epi-Quant,针对图像分割、细胞分类、细胞追踪等核心模块进行深度优化,不仅能够缩短整体计算时间,提高大规模数据处理能力,还能够扩展 Epi-Quant 在复杂实验场景中的适用性与竞争力,进一步推动细胞动力学领域的研究进展,提升生物医学基础研究与临床应用的效率与深度。
\subsection{国内外研究现状}

国内外在细胞图像并行处理领域的研究正呈现出蓬勃发展的态势,其核心目标在于借助高性能计算资源,有效加速对细胞图像的分析进程,以应对日益增长的数据量和分析复杂性。\cite{ref7}目前,国内外的研究均在积极探索前沿技术与方法,并取得了一定的进展。

在国际上,多线程技术已被广泛应用于细胞图像分析的各个环节。例如,HiTIPS平台集成了先进的图像处理和机器学习算法,实现了细胞和细胞核的自动分割、信号检测和追踪等功能,显著提升了大规模图像数据的处理效率\cite{ref8}。 此外,Cecelia工具箱通过整合多个开源软件包,提供了一个统一的数据管理平台,使非专业人员也能进行定量的多维图像分析,推动了多线程技术在图像分析中的普及\cite{ref9}。 在算法层面,Goldberg等人开发的增量广度优先搜索算法在图像分割中表现出更高的速度和鲁棒性,展示了多线程算法在提升图像处理性能方面的潜力\cite{ref10}。 

在国内,研究者们也在积极探索多线程技术在细胞图像处理中的应用。清华大学戴琼海教授团队开发的DeepCAD-RT系统是国内在该领域的代表性成果之一。该系统通过引入多线程调度机制,实现了与显微镜硬件的深度融合,显著提升了荧光图像的实时处理能力。在神经元钙成像、免疫细胞迁移等多种生命过程的观测任务中,DeepCAD-RT展现出强大的去噪性能,推动了高通量细胞图像分析的实时化进程\cite{ref11}。
这些研究成果表明,国内在多线程图像处理技术方面取得了积极进展,尤其在提高图像处理效率和精度方面展现出显著优势。未来,随着计算资源的进一步优化和算法的持续改进,多线程技术将在细胞图像处理领域发挥更加重要的作用。

综上所述,国内外在多线程技术应用于细胞图像并行处理方面均取得了显著成果。国际上,研究重点集中在工具平台的开发和算法优化上,而国内则在具体应用和方法实现方面取得了实质性进展。未来,随着图像数据规模的进一步扩大和分析需求的多样化,结合多线程技术与人工智能、深度学习等先进方法,将为细胞图像分析提供更强大的技术支持,推动生物医学研究的深入发展。
\subsection{研究内容}
本文的主要研究内容是在 Epi-Quant 平台中应用 Python 并行计算技术,以提高细胞图像分析任务的计算效率。首先,本文将详细介绍 Python 并行计算的相关原理,包括多线程和多进程的工作机制。接着,介绍 Epi-Quant 平台的整体流程,重点讲解细胞特征提取、细胞分类和细胞追踪的计算步骤,并进行详细的耗时分析,为后续并行化设计提供依据。然后,本文着重于 Epi-Quant 的并行化设计,首先进行了可行性分析,指出哪些任务适合并行化,接着对细胞特征提取、细胞分类和细胞追踪部分的并行化进行了详细设计,并讨论了任务拆分和负载均衡问题,确保并行计算的高效性和正确性。第五章展示了并行计算实验的结果,包括实验目的、实验环境、实验数据、实验参数设置以及结果展示等内容。重点分析了各部分任务的加速比和并行效率,并对结果进行了深入讨论。最后,第六章总结了本文的研究成果,并对未来在细胞图像分析领域应用并行计算的研究方向进行了展望和讨论。
\section{Epi-Quant平台简述}
\subsection{流程简述}
Epi-Quant 平台利用 Cellpose 实现高效的图像分割功能,为细胞形态的精确分析提供基础。Cellpose 是一种强大的深度学习工具,能够自动识别和分割显微镜图像中的细胞结构。在初次使用时,Epi-Quant 会通过 Cellpose 处理输入的细胞图像,并生成包含分割信息的 npy 文件作为后续分析的基础。

在完成图像分割后,Epi-Quant 采用主成分分析法(PCA)与 K-means 聚类算法对细胞进行分类。首先,PCA 被用来提取细胞形态的主要特征,简化数据维度的同时保留关键信息。然后,通过 K-means 聚类算法根据这些特征对细胞进行分组。分类完成后,Epi-Quant 会生成逐帧细胞分类结果的可视化图,并输出相关数据文件,包括逐帧细胞分类结果文件、PCA 主成分贡献度文件等。

为了实现精准的细胞追踪,Epi-Quant 计算细胞间的差异度矩阵以评估相似性,并结合匈牙利算法来确定同一细胞在不同时间点的最佳匹配。这种方法确保了细胞追踪过程的高效性和准确性。此外,Epi-Quant 还能生成细胞位移矢量图和均方位移(Mean Squared Displacement, MSD)箱型图,有助于深入理解细胞的运动模式和扩散行为。通过这一系列步骤,Epi-Quant 提供了一套完整的工具集,用于从图像分割到细胞动态变化的全面分析。

\begin{figure}[htbp]
    \centering
    \begin{tikzpicture}[node distance=2cm] % 增加节点间的距离

        \node (start) [startstop] {开始};
        \node (cellFeature) [process, below of=start] {细胞特征提取};
        \node (decision1) [decision, below of=cellFeature] {是否需要细胞分类?}; 
        \node (cellClassify) [process, below of=decision1] {细胞分类};
        \node (check1) [decision, below of=cellClassify] {全部分类正确?};
        \node (fixClassify) [process, left of=check1, xshift=-2.5cm] {细胞分类结果纠错}; 
        \node (trackCell) [process, below of=check1] {细胞追踪};
        \node (check2) [decision, below of=trackCell] {全部追踪正确?};
        \node (fixTrack) [process, left of=check2, xshift=-2.5cm] {细胞追踪结果纠错};
        \node (dataGen) [process, below of=check2] {定量分析数据生成};
        \node (display) [process, below of=dataGen] {定量分析结果展示};
        \node (end) [startstop, below of=display] {结束};
        
        % 绘制箭头和连接线
        \draw [arrow] (start) -- (cellFeature);
        \draw [arrow] (cellFeature) -- (decision1);
        \draw [arrow] (decision1) -- node[anchor=east] {是} (cellClassify);
        \draw [arrow] (cellClassify) -- (check1);
        \draw [arrow] (check1) -- node[anchor=east] {否} (fixClassify);
        \draw [arrow] (fixClassify.west) -- ++(-0.5,0) |- (cellClassify.west); 
        \draw [arrow] (check1) -- node[anchor=east] {是} (trackCell);
        \draw [arrow] (decision1.east) -- ++(0.5,0) node[anchor=north] {否} |- (trackCell.east); 
        \draw [arrow] (trackCell) -- (check2);
        \draw [arrow] (check2) -- node[anchor=east] {否} (fixTrack);
        \draw [arrow] (fixTrack.west) -- ++(-0.5,0) |- (trackCell.west);
        \draw [arrow] (check2) -- node[anchor=east] {是} (dataGen);
        \draw [arrow] (dataGen) -- (display);
        \draw [arrow] (display) -- (end);
        
    \end{tikzpicture}
    \caption{Epi-Quant平台工作流程}\label{fig:flowchart2_1}
\end{figure}
\newpage
\subsection{耗时分析}
为了全面评估细胞特征提取、细胞分类和细胞追踪三个模块的性能表现,本文分别对每个流程进行了五次独立测试,并取其平均值作为最终结果。这一方法能够有效减小由于系统环境波动或随机因素带来的误差,提高实验数据的稳定性与可靠性,从而为后续分析提供更具代表性的参考依据。同时,多次实验也有助于观察各阶段运行时间的一致性,验证程序执行的稳定性,增强整体实验结论的科学性与说服力。将所得数据列为下表:
\begin{table}[!htbp]
    \centering
    \caption{实验时间统计}
    \begin{tabular*}{350pt}{@{\extracolsep{\fill}}cccc}
        \toprule
        实验次数 & 细胞特征提取(秒) & 细胞分类(秒) & 细胞追踪(秒) \\
        \midrule
        第1次   & 28.8228 & 45.9401 & 114.5828 \\
        第2次   & 25.8424 & 41.7089 & 106.1937 \\
        第3次   & 24.9972 & 41.3358 & 106.9872 \\
        第4次   & 26.5087 & 41.5460 & 104.9517 \\
        第5次   & 27.0539 & 43.1250 & 107.0588 \\
        平均值   & 26.6450 & 42.7312 & 107.9548 \\
        \bottomrule
    \end{tabular*}
\end{table}

可以看出,在串行的程序设计下,各部分耗时均有着较大的提升空间。

对于细胞特征提取部分,将代码中的各个部分时间作进一步的记录,随机选取执行一次细胞特征提取任务,将时间记录为下表:

\begin{table}[!htbp]
    \centering
    \caption{细胞特征提取-各步骤耗时统计}
    \begin{tabular*}{350pt}{@{\extracolsep{\fill}}ccc}
        \toprule
        步骤             & 耗时(秒) & 占比(\%) \\
        \midrule
        文件排序时间     & 0.0003     & 0.001      \\
        数据加载时间     & 0.5266     & 2.007      \\
        UI 更新时间       & 0.0717     & 0.273      \\
        特征计算时间     & 24.7285    & 94.251     \\
        DataFrame 合并时间 & 0.0276    & 0.105      \\
        Excel 保存时间   & 0.6098     & 2.323      \\
        总执行时间       & 26.2361    & 100.000    \\
        \bottomrule
    \end{tabular*}
\end{table}

可以看出,细胞特征提取时间中占比部分大的主要是特征提取部分,涉及到对于细胞面积以及其他形态学特征的计算,串行会让计算过程变得较为缓慢,本文将代码运行改进的重点放在特征提取的部分,通过并行技术将计算效率进一步提升,达到理想的效果,为后续该工具的拓展应用奠定基础。

对于细胞分类部分,将代码中的各个部分时间作进一步的记录,随机选取执行一次细胞分类任务,将时间记录为下表:

\begin{table}[!htbp]
    \centering
    \caption{细胞分类-各步骤耗时统计}
    \begin{tabular*}{300pt}{@{\extracolsep{\fill}}ccc}
        \toprule
        步骤               & 耗时(秒) & 占比(\%) \\
        \midrule
        PCA 执行耗时       & 1.3139     & 2.735      \\
        K-means 执行耗时   & 1.6279     & 3.389      \\
        图像处理耗时       & 44.9853    & 93.646     \\
        总执行耗时         & 48.0328    & 100.000    \\
        \bottomrule
    \end{tabular*}
\end{table}
根据细胞分类过程中各阶段的耗时分析,图像处理作为整个流程中最耗时的部分主要集中在update\_image函数中。该函数执行了多项任务,包括从 .npy 文件加载数据、依据帧号匹配并加载相应的图像文件、对图像进行预处理(如转换为RGB格式和数据类型调整)、基于聚类结果对不同类型的细胞进行着色、计算相邻细胞间的邻接性以及使用KDTree算法计算细胞中心到边界点的最短距离等。具体来说,在一次运行中,update\_image函数总共耗时约44.9853秒,占总执行时间(48.0328秒)的约93.646\%

这表明优化此函数中的操作,特别是那些计算密集型的任务(例如图像预处理和邻接性分析),将显著减少总体执行时间,进而提高细胞分类的整体效率。通过采用适当的并行化策略,如多进程处理,可以有效加速这些步骤,从而大幅缩短细胞分类所需的时间。

对于细胞追踪部分,将代码中的各个部分时间作进一步的记录,随机选取执行一次细胞追踪任务,将时间记录为下表:

\begin{table}[!htbp]
    \centering
    \caption{细胞追踪-各步骤耗时统计}
    \begin{tabular*}{300pt}{@{\extracolsep{\fill}}ccc}
        \toprule
        步骤               & 时间 (秒)  & 占比 (\%) \\
        \midrule
        追踪所有细胞       & 77.5437    & 71.64     \\
        保存追踪数据       & 0.0769     & 0.07      \\
        生成对比图像       & 30.6142    & 28.29     \\
        总执行耗时         & 108.2348   & 100.00    \\
        \bottomrule
    \end{tabular*}
\end{table}
run\_cell\_tracking 是整个细胞追踪流程中的核心函数,负责对给定的细胞信息(如位置、形态特征等)进行逐帧匹配与追踪。该函数首先根据细胞类型(如羊浆膜细胞或表皮细胞)划分处理逻辑,并为每一帧图像提取对应的细胞数据。随后,它通过计算当前帧与下一帧中细胞之间的多维特征距离(如面积、长宽比、中心坐标等),使用匈牙利算法(linear\_sum\_assignment)实现最优匹配。最终,将所有帧间的匹配结果进行合并,生成完整的细胞运动轨迹。
函数中最关键的部分是逐帧匹配细胞关系并生成追踪路径。这部分不仅决定了细胞在时间序列上的连续性,也直接影响了追踪的准确性和鲁棒性。尤其是利用多个细胞特征加权后的欧氏距离作为匹配依据,并结合匈牙利算法求解最小代价匹配,使得系统能够有效应对细胞分裂、移动、遮挡等复杂情况。
由于每一组相邻帧之间的匹配操作是相对独立的(即 Frame i 与 i+1 的匹配不影响 Frame j 与 j+1 的匹配,只要没有跨帧依赖),因此可以将帧间匹配部分(即主循环内的内容)拆分为多个子任务,通过多线程或多进程方式进行并行处理。这样可以显著提升大规模数据集下的运行效率,缩短整体追踪耗时。
