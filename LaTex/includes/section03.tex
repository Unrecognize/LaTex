\section{Epi-Quant并行化设计}
\subsection{可行性分析}
在 Epi-Quant 中,细胞特征计算、细胞分类的细胞轮廓提取,以及细胞追踪中的帧间匹配任务,都需要处理大量计算数据。随着数据集的扩大,传统的串行计算方式在执行这些任务时逐渐变得不够高效,特别是当数据量非常大时,处理时间会显著增加,影响整体分析效率。因此,采用并行计算来优化这些步骤,不仅可以提升计算速度,还能更好地利用现有的计算资源\cite{ref14}。
首先,细胞特征计算是整个工具包中的核心任务之一,涉及从细胞轮廓中提取各种形态学特征(比如面积、周长等)。如果按传统串行方式,每次只能处理一个细胞,计算进度会比较缓慢。而引入并行计算后,细胞的特征提取可以分配给不同的计算单元,彼此独立地处理每个细胞,这样可以显著加快整个计算的速度。
其次,细胞分类过程中,轮廓提取同样是关键。传统的方法一般需要逐图像处理,这对于大数据集来说极其低效。通过并行计算,可以把多个图像或者细胞的轮廓提取任务分配到多个计算单元上并行处理,从而加快整个过程。在处理具有较大形态差异的细胞时,并行计算特别有效,因为它能够快速处理不同细胞的特征,从而保持分类的准确性和效率。
最后,细胞追踪中的帧间匹配是一个具有挑战性的任务,涉及到时间序列的匹配。由于每帧的匹配任务独立性较强,因此非常适合并行计算。通过将帧间的匹配任务拆解,多个计算单元可以同时执行任务,这样不仅提升了匹配速度,也保证了长时间追踪过程中动态变化的及时捕捉。
并行计算的引入,使得 Epi-Quant 在图像处理中的整体效率得到了显著提升。通过将计算任务拆解并分配到多个处理单元,可以更好地利用 CPU 或 GPU 的多核能力。在处理大规模数据时,系统能够更智能地分配任务,避免某些计算单元负荷过重,从而最大化计算效率。而且并行计算大大降低了串行计算中的瓶颈,能加快数据处理的速度,进一步缩短了整体计算的时间。
\subsection{并行程序设计}
\subsubsection{细胞特征提取部分的并行化}
细胞特征计算步骤采用了基于线程池的并行处理策略,以提高细胞信息提取的效率。具体而言,在对每个输入的 .npy 文件所包含的细胞区域进行处理时,系统将所有独立细胞(由 unique\_ids 标识)分配至多个并发线程中,每个线程独立执行一个细胞的特征计算任务。该任务包括细胞边界识别、面积与周长计算、最大水平与垂直长度分析、拟合椭圆参数提取、多边形近似顶点数统计、以及圆度与P值等形态学特征的推导。通过 ThreadPoolExecutor 实现的多线程机制,多个细胞的特征计算过程得以并发执行,从而显著缩短了整体数据处理的时间,提升了系统的响应速度与运算效率。各线程完成计算后,其结果被统一收集至共享列表中,为后续的数据整合与输出提供支持。
\subsubsection{细胞分类部分的并行化}
在细胞分类过程中,并行处理被用于加速对大量细胞轮廓信息的分析和处理。具体来说,当一个新帧图像被加载后,系统首先识别出该帧中所有独特的细胞ID。针对每个细胞ID,系统会启动一个独立的任务或线程,这些任务并发地执行细胞处理函数Process\_Cell。在此函数中,每个细胞根据其聚类标签(如大细胞或小细胞)被分配特定的颜色标记,并且该细胞的轮廓点被提取并保存到相应的集合中(例如,蓝色细胞轮廓集或黄色细胞轮廓集)。通过利用线程池来管理这些并发任务,程序可以同时处理多个细胞的数据,从而显著提升整体处理效率。一旦所有线程完成其执行,后续步骤包括计算相邻细胞之间的最短距离等进一步分析也会基于这些初步结果进行。这种并行策略确保了即使面对高通量数据时,细胞分类和分析过程也能高效、准确地完成
\subsubsection{细胞追踪部分的并行化}
首先,程序提取所有有效帧号,并通过进程池机制(ProcessPoolExecutor)对相邻帧之间的细胞配对任务并行处理。每个进程独立完成一组帧之间的细胞对应计算,去除异常匹配并输出结果文件。匹配过程基于几何特征差异,确保配对的准确性。

随后,采用线程池(ThreadPoolExecutor)对生成的匹配文件并行读取。程序依次将各帧之间的匹配结果组合成跨帧追踪表,逐步构建细胞在时序图像中的连续轨迹。最终,整合结果被统一保存为 Excel 文件。

通过在不同阶段采用进程与线程并行机制,此方法在保证追踪精度的同时,显著缩短了处理大批量图像数据的时间。

\begin{figure}[htbp]
    \centering
    \begin{algorithm}[H]
        \caption{多线程细胞信息提取算法}
        \label{alg:cell_extraction}

        \textbf{输入:} 包含细胞轮廓的 .npy 文件路径 \\
        \textbf{输出:} 提取后的细胞特征数据(Excel 文件)
        
        \begin{algorithmic}[1]
            \STATE 获取所有 .npy 文件并按编号排序
            \FOR {每个文件}
                \STATE 加载图像数据并获取细胞 ID 列表
                \STATE 初始化用于存储结果的列表
                
                \STATE 启动线程池
                \FORALL {每个细胞 ID}
                    \STATE 计算细胞边界点坐标
                    \IF {排除边缘细胞且该细胞位于边缘}
                        \STATE 跳过该细胞
                    \ELSE
                        \STATE 填充轮廓区域并计算面积
                        \STATE 查找轮廓并计算周长
                        \STATE 拟合椭圆,获取长短轴及角度
                        \STATE 计算最小外接圆和最大内接圆半径
                        \STATE 计算圆度和 P 值
                        \STATE 收集细胞中心坐标与边界信息
                        \STATE 将结果组织为 DataFrame 并返回
                    \ENDIF
                \ENDFOR
                
                \STATE 收集所有线程的结果并合并
            \ENDFOR
            
            \STATE 将最终结果保存为 Excel 文件
        \end{algorithmic}
    \end{algorithm}
    \caption{多线程细胞信息提取算法流程}\label{fig:flowchart3_2}
\end{figure}
\begin{figure}[htbp]
    \centering
    \begin{algorithm}[H]
        \caption{多线程细胞分类算法流程}
        \label{alg:cell_classification}

        \textbf{输入:} 标注信息文件路径、图像路径、分类结果、输出目录 \\
        \textbf{输出:} 分类后并标注的图像结果
        
        \begin{algorithmic}[1]
            \STATE 读取当前帧的标注文件并提取相关数据
            \IF {图像未嵌入标注文件}
                \STATE 自动查找并载入对应图像
            \ENDIF
            
            \STATE 初始化用于存储不同类别轮廓的容器
            
            \STATE 提取所有单元编号
            
            \STATE 启动线程池
            \FORALL {每个单元编号}
                \STATE 提交处理任务以判断类别并提取轮廓
                \IF {编号为背景或无效}
                    \STATE 跳过
                \ELSIF {属于第一类}
                    \STATE 存储轮廓至对应容器
                    \STATE 着色并可视化该区域
                \ELSIF {属于第二类}
                    \STATE 同上操作,使用不同颜色
                \ENDIF
            \ENDFOR
            
            \STATE 等待所有线程完成处理
            
            \STATE 后处理分析:寻找配对关系与几何关系
            \STATE 计算指定边缘之间的距离
            \STATE 展示并保存结果图像
        \end{algorithmic}
    \end{algorithm}
    \caption{多线程细胞分类算法流程}\label{fig:flowchart_classification}
\end{figure}
\begin{figure}[htbp]
    \centering
    \begin{algorithm}[H]
        \caption{多进程细胞追踪算法}
        \label{alg:cell_tracking}

        \textbf{输入:} 所有帧的细胞信息 cells\_info,进度范围 progress\_start 至 progress\_end \\
        \textbf{输出:} 合并后的细胞追踪表格(Excel 文件)

        \begin{algorithmic}[1]
            \STATE 统一 cells\_info 的索引为字符串格式
            \STATE 根据 leading\_edge 字段内容,确定细胞类型与输出路径
            \IF{需要读取已有细胞追踪结果}
                \STATE 获取现有结果中最大编号,作为新的起始索引
            \ENDIF
            \STATE 初始化输出目录,创建输出路径

            \STATE 提取所有帧编号 fig\_ids,并排序
            \IF{帧编号为空}
                \STATE 报错并终止
            \ENDIF

            \STATE 初始化 all\_matching\_results 列表

            \COMMENT{第一阶段:使用多进程并行计算帧间匹配}
            \STATE 启动进程池
            \FOR{每个 fig\_id}
                \STATE 将匹配任务提交给进程池执行 parallel\_frame\_processing(fig\_id)
            \ENDFOR
            \FOR{每个完成的任务}
                \STATE 获取 fig\_id 与匹配结果 matching\_result
                \STATE 为结果添加唯一编号
                \STATE 去除匹配误差等于最大阈值的行
                \STATE 保存匹配结果至对应 Excel 文件
                \STATE 累加结果并更新索引
            \ENDFOR

            \COMMENT{第二阶段:主线程更新进度条}
            \FOR{每个 fig\_id}
                \STATE 根据已完成帧数更新进度条显示
            \ENDFOR

            \COMMENT{第三阶段:使用多线程读取所有匹配文件并整合}
            \STATE 获取所有匹配结果文件
            \STATE 启动线程池
            \FOR{每个文件名}
                \STATE 并行读取并解析文件内容为 DataFrame
            \ENDFOR
            \STATE 按帧序整合结果,构建完整追踪链

            \STATE 将最终结果保存为 Excel 文件
        \end{algorithmic}
    \end{algorithm}
    \caption{多进程细胞追踪算法流程}\label{fig:flowchart_tracking}
\end{figure}
